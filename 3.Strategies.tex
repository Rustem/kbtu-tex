\chapter{Team-motivation strategies and personal productivity}

They way to create an intention to develop / tailor appropriate methodology is to begin with personal productivity and motivation behind doing the project. Why?

Remind the problem of motivation, Dean Spitzer's report. Is it appropriate for startups?

\section{``Action Method" application and concept behind}

The Action Method begins with a simple premise: everything is a project. This applies not only to the big presentation on Wednesday or the new campaign you’re preparing, but also to the stuff you do to advance your career (a “career development” project), or to employee development (each of your subordinates represents a single “project” in which you keep track of performance and the steps you plan to take to help him or her develop as an employee). Managing your finances is a project, as is doing your taxes or arranging the upcoming house move.Like most creative people, I’m sure you struggle to make progress in all of your projects, with the greatest challenge being the sheer number of projects before you! But once you have everything classified as a project, you can start breaking each one down into its primary components: Action Steps, References, and Backburner Items.


Every project in life can be reduced into these three primary components.

Action Steps

are the specific, concrete tasks that inch you forward: redraft and send the memo, post the blog entry, pay the electricity bill, etc.

References

are any project-related handouts, sketches, notes, meeting minutes, manuals, websites, or ongoing discussions that you may want to refer back to. It is important to note that references are not actionable—they are simply there for reference when focusing on any particular project. Finally, there are

Backburner Items

—things that are not actionable now but may be someday. Perhaps it is an idea for a client for which there is no budget yet. Or maybe it is something you intend to do in a particular project at an unforeseen time in the future.

Every project in life can be reduced into these three primary components.

Let’s consider a sample project for a client. Imagine a folder with that client’s name on it. Inside the folder you would have a lot of References—perhaps a copy of the contract, notes from meetings, and background information on the client. The Action Steps—the stuff you need to do—could be written as a list, attached to the front of the folder. And then, perhaps on a sheet stapled to the inside back cover of the folder, your Backburner list could keep track of the non-actionable ideas that come up while working on the project—the stuff you may want to do in the future.

With this hypothetical folder in mind, you can imagine that the majority of your focus would be on the Action Steps visible on the front cover. These Action Steps are always in plain view. They catch your eye every time you glance at the project folder. And, as you review all of your project folders every day, what you’re really doing is just glancing over all of the pending Action Steps.

We call it the “Action Method” because it helps us live and work with a bias toward action. The actionable aspects of every project pop out at us, and the other components are organized enough to provide peace of mind while not getting in the way of taking action.

Personal projects can also be broken down into the same three elements. If you take some time to look around your desk, you might find some notes or reminders that you’ve left for yourself. Perhaps you see a household bill that requires payment (an Action Step in the project “Household Management”), or a copy of your car insurance certificate (a Reference in the project “Insurance”). Maybe it is a cutout of a great vacation spot you want to visit someday (a Backburner Item in the project “Vacation Planning”).

Consider a few projects in your life—some work-related and some personal. The components of these projects are either in your head or all around you—sentences in emails, sketches in notebooks, and scribbles on Post-it notes. The Action Method starts by considering everything around you with a project lens and then breaking it down.

The Action Method starts by considering everything around you with a project lens and then breaking it down.

Perhaps you have an idea for a screenplay that you’d like to write someday. If so, make it a “Backburner Item” in the “New Screenplay Ideas” project or perhaps in a more general “Bold Ideas” project that you may review only a couple of times every year. (In ActionMethod Online, “Backburner Items” can be captured by creating an Action Step without a due date, which will automatically be displayed at the bottom of each project.) While some projects realistically won’t get much of your focus, they will help store the Backburner Items and References that you generate.

Of course, your hope is that someday a few of these Backburner Items will be converted into real Action Steps—which will, in turn, lead to a new and more active project, like your screenplay. Action Steps are the building blocks of accomplishment. But sometimes, at certain periods of life, you can’t afford to take certain actions. For this reason, it is okay to have dormant projects filled with References and Backburner Items. The time will come when some of these projects return to the surface with some Action Steps.

As you go about your day, you should think in terms of which project is associated with what you are doing at any point in time. Whether in a meeting, brainstorming session, chance conversation, article, dream, or eureka moment in the shower, you are generating Action Steps, References, and Backburner Items at a fast clip. Everything is associated with a project. Sadly, much of this output will be lost unless you capture it and assign it properly.

In the sections ahead, we will explore the three primary components of projects in more detail and how they should be managed. But the key realization should be that everything in life is a project, and every project must be broken down into Action Steps, References, and Backburner Items. It’s that simple.

The key realization is that everything in life is a project, and every project must be broken down into Action Steps, References, and Backburner Items.

Of course, in the digital era, information comes to us in many forms. Projects are not always kept in folders. In fact, projects are managed across many mediums. And the components of projects come to us in the form of emails, status updates, files as downloads, and a barrage of links that we save daily. Nevertheless, the Action Method still applies; everything belongs to a project. With the Action Method in mind, we can make better use of online and offline tools that organize information.


Action Steps are the most important components of projects—the oxygen for keeping projects alive. No Action Steps, no action, no results. The actual outcome of any idea is dependent on the Actions Steps that are captured and then completed by you or delegated to someone else. Action Steps are to be revered and treated as sacred in any project.The more clear and concrete an Action Step is, the less friction you will encounter trying to do it. If an Action Step is vague or complicated, you will probably skip over it to others on your list that are more straightforward.

To avoid this, start each Action Step with a verb:

   * Call programmer to discuss . . .
   * Install new software for . . .
   * Research the possibility of . . .
   * Mock up a sample of the . . .
   * Update XYZ document for . . .

Verbs help pull us into our Action Steps at first glance, efficiently indicating what type of action is required. For similar reasons, Action Steps should be kept short.

The more clear and concrete an Action Step is, the less friction you will
encounter trying to do it.

Imagine you and I are having a conversation in a meeting. I describe to you what I want to accomplish and show you some diagrams that further describe the idea. You reply by saying, “I see what you’re trying to do. There’s a guy I know who designed a great website with the same type of functionality.” Upon saying this, I record an Action Step to follow up with you regarding that website:

   * Follow up with [name] re: guy’s website w/ similar functionality.


A colleague might say, “Let’s revisit that old draft and consider the initial plan that we had—maybe it was better? Let me know what you
think.” In that case, your Action Step would be:

   * Print out old draft, follow up with [colleague’s name] re: alternative plan.

Sometimes you will find yourself waiting for a response to an email or a phone call. It is easy to forget something when it is in someone else’s court! To trigger yourself to follow up if you don’t hear back, you may want to create a separate Action Step.

Action Steps arise from every idea exchange. Even the smallest of Action Steps, when captured, will make a big difference because they create momentum. A missed Action Step can cause miscommunications, more meetings, and could be the difference between success and failure in any project.

Here are some key practices:

Capture Action Steps everywhere. Ideas don’t reveal themselves only in meetings, and neither should Action Steps. Ideas come up when you are reading an article, taking a shower, daydreaming, or getting ready for bed. If you think of someone that you met with a month ago regarding a certain project but have not yet followed up with, create an Action Step to “follow up with XYZ regarding . . .” If you are opening your mail and come across a wedding invitation, your Action Step is to RSVP.

Think of Action Steps expansively—as anything you might want to do—and capture all of them, not only the ones that arise during meetings.

Ideas don’t reveal themselves only in meetings, and neither should Action Steps.

Having some sort of pad or recording device handy will enable you to capture actions as they come to mind. Our team developed the iPhone version of Action Method Online because users wanted a quick and “anytime, anywhere” way to capture Action Steps and assign them to a project. Whatever medium you choose to use for capturing Action Steps, it should always be readily available. Your system should also make it easy to return to your Action Steps at a later time and distinctly recall what you were thinking. And, most important, you must always be able to distinguish Action Steps from References—the regular notes and non-actionable ideas that you may have also written down.

An unowned Action Step will never be taken.

Every Action Step must be owned by a single person. While some Action Steps may involve the input of different people, accountability must reside in one individual’s hands at the end of the day. Some people who lead teams or have assistants will capture Action Steps and delegate them to others. However, even when the onus to complete an Action Step has been delegated to someone else, the Action Step must still be owned by the person ultimately responsible.

Every Action Step must be owned by a single person.

The reason comes down to accountability. The practice of simply emailing someone a task to complete does not provide any assurance that it will be completed. For this reason, Action Steps that you are ultimately responsible for should remain on your list until completed—even when you have delegated them to others. Simply marking that the Action Step has been delegated and to whom is sufficient:

   * Print out old draft, follow up with Alex re: other plan (Oscar is doing).

Treat managerial Action Steps differently.

Aside from the Action Steps that you and only you can do, there are three other types of Action Steps you should keep in mind as the leader of a project. The first type is delegated Action Steps, which we just discussed above. The second type is “Ensure Action Steps.” Sometimes you will want to create an Action Step to ensure that something is completed properly in the future. Rather than being a nag to your team, you can create an Action Step that starts with the word “ensure.” For example, “Ensure that Dave updated the article with the new title.” If you use a digital tool to manage your Action Steps, you can always search by the word “ensure” (to only view Action Steps that start with “ensure”) and spend some time verifying that these items have been done. Creating “Ensure Action Steps” is a better alternative then sending numerous reminder emails to your team when you are worried about something slipping through the cracks.

The last type of managerial Action Step is the “Awaiting” Action Step. When you leave a voicemail for someone, send a message to a potential customer, or respond to an email and clear it from your inbox, you’re liable to forget to follow-up if the person fails to respond. By creating an Action Step that starts with “Awaiting,” you can keep track of every ball that is out of your court. When I respond via email to a potential client, I create an Action Step like “Awaiting confirmation from Joe at Apple re: consultation,” saved in the project “Consulting Work.” In my online task manager I will set a target date for one week later. After a week passes, I will be reminded to follow up. Sometimes I will search all my Action Steps, across projects, with the word “awaiting” and dedicate an hour to follow up on everything.

Foster an action-oriented culture.

Your team needs an action-oriented culture to capitalize on creativity. It may feel burdensome or even a bit aggressive to ask people to capture an Action Step on paper, but fostering a culture in which such reminders are welcome helps ensure that Action Steps are not lost. Some of the most productive teams I have observed are comfortable making sure that others are capturing Action Steps. Aside from friendly questioning along the lines of “Did you capture that?” some teams take a few minutes at the end of every meeting to go around the table and allow each person to recite the Action Steps that he or she captured. Doing so will almost always reveal a missed Action Step or a duplication on two people’s lists. This simple practice can save time and prevent situations in which, weeks later, people are wondering who was doing what or how something got lost in the shuffle.

Your team needs an action-oriented culture to capitalize on creativity.

Attraction breeds loyalty.

When it comes to the mechanics of capturing action steps, you should find the solution that fits you best. Keep in mind that the design of your productivity tools will affect how eager people are to use them. Attraction often breeds commitment: if you enjoy your method for staying organized, you are more likely to use it consistently over time. For this reason, little details like the colors of folders you use or the quality of the paper can actually help boost your productivity.

In her book The Substance of Style, journalist Virginia Postrel shares an anecdote about usability guru Donald Norman’s assertion that “attractive things work better.” When the first color computer monitors became available commercially, Norman wanted to justify the value of buying the expensive monitors instead of the standard black-and-white displays. Nowadays, this decision might seem obvious, but back in the day before the World Wide Web and color printers, the value of a color monitor for functions like word processing was unproven.

“I got myself a color display and took it home for a week,” Norman recalled. “When the week was over, I had two findings. The first finding was that I was right, there was absolutely no advantage to color. The second was that I was not going to give it up.” In her analysis of Norman’s findings, Postrel explains, “The difference lay not in ‘information processing’ but in ‘affect,’ in how full-color monitors made people feel about their work.” In other words, the aesthetics of the tools you use to make ideas happen matter.


As you move through your day of meetings, brainstorms, and other occasions of creativity, you will start to accumulate Action Steps, References, and Backburner Items. Handouts, random pages of notes, emails, and social network messages will build up all around you. Often these items will get buried in notebooks, pockets, online inboxes, and computer files almost as soon as they are created or received. Ideally, in your written notes you will have kept your

Action Steps separate from everything else. However, you will still need time for processing – going through all of your day’s notes and communications, and distilling them all down to the primary elements. For those who still take paper notes and appreciate tangible project management, you will want to use a tangible inbox—a general pile of stuff that has yet to be classified. Most productivity frameworks—like David Allen’s Getting Things Done—suggest such a central clearinghouse for all of the stuff that you accumulate but can’t immediately execute or file. This inbox is not a final destination, but rather a transit terminal where items await processing. During a busy day of meetings, you will not have time to start taking action or filing things away.

How about all of the digital stuff that flows in every day? Your email inbox is the primary landing spot, but information also flows into other online applications. While your tangible inbox, sitting on your desk, is singular, the digital equivalent is becoming more of a collective. Ideally, you should set your social network profiles to forward messages to your email inbox for the sake of aggregation. When you commit time for processing, you’ll want to limit the number of places you need to visit.

Ideally, you should set your social network profiles to forward messages to your email inbox for the sake of aggregation.

If you can’t aggregate the flow of emails and other digital communications in the same place, then you need to define the various pieces of your collective digital inbox. For example, my collective digital inbox includes my email program (which receives messages from all other networks), a Twitter aggregator, and the inbox in my task management application (where I accept/reject stuff sent from my colleagues who use the same application – and then manage this information by project). When the time comes for processing, these are the three digital places I need to visit, along with the tangible inbox full of papers on my desk.

As you can see, the “inbox” of the 21st century varies for everyone. You must concretely define your collective inbox before you start processing. Peace of mind and productivity starts when you know where everything is. The combined inbox says, “Don’t worry, all of your stuff (and the Action Steps, Backburner Items, and References contained within) are in a defined place, waiting for you and ready to be sorted.”

Peace of mind and productivity starts when you know where everything is.

If you live a digital lifestyle, your ability to process your inbox may be at particular risk without some sense of discipline. The reason: in the era of mobile devices and constant connectivity, it has become all too easy for others to send us messages. As such, our ability to control our focus is often crippled by the never-ending flow of incoming phone calls, emails, text messages, and in-person interruptions—not to mention messages from other online services. Thus it is important that you avoid the trap of what I have come to call “reactionary workflow.”

The state of reactionary workflow occurs when you get stuck simply reacting to whatever flows into the top of an inbox. Instead of focusing on what is most important and actionable, you spend too much time just trying to stay afloat. Reactionary workflow prevents you from being more proactive with your energy. The act of processing requires discipline and imposing some blockades around your focus. For this reason, many leaders perform their processing at night or at a time when the flow dies down.

Time spent processing is arguably the most valuable and productive time of your day. While processing, you will sort everything and distinguish Action Steps, Backburner Items, and References. With Action Steps, you will decide what can be done quickly and what must be tracked over time by project—and possibly delegated. With other materials, you will make judgments about what can be thrown away and what must be filed.

As you start to tackle your collective inbox, you will realize that any inbox, on its own, is a pretty bad action management tool. It is difficult to keep your Action Steps separate from References and other noise. The constant stream of email certainly doesn’t help. In addition to email, you may also receive other types of incoming communications in the form of Tweets, Facebook messages, etc. Some are actionable, or contain actionable elements, while others are simply for reference (or for fun).

Time spent processing is arguably the most valuable and productive time of your day.

Given the unyielding flow of communications, you will want to capture and manage your Action Steps separately. Despite the many tricks involving “action subfolders” and other ways to manage and prioritize Action Steps within an email system, there is nothing better than giving Action Steps their own sacred space to be managed by project.

The Action Method suggests that Action Steps should be managed separately from communications. The solution can be as simple as a spreadsheet or to-do list where all Action Steps are tracked (and can be sorted by project name or due date). You can also make use of more advanced project management applications that manage Action Steps and support delegation and collaboration. What you want to avoid is a mishmash of actionable items amidst hundreds of verbose emails and other messages scattered in various places.

\section{Gamification}

Following the success of the location-based service Foursquare, the idea of using game design elements in non-game contexts to motivate and increase user activity and retention has rapidly gained traction in interaction design and digital marketing. Under the moniker ``gamification", this idea is spawning an intense public debate as well as numerous applications – ranging across productivity, finance, health, education, sustainability, as well as news and entertainment media. Several vendors now offer ``gamification" as a software service layer of reward and reputation systems with points, badges, levels and leader boards.

This commercial deployment of `gamified' applications to large audiences potentially promises new, interesting lines of inquiry and data sources for human-computer interaction (HCI) and game studies – and indeed, ``gamification" is increasingly catching the attention of researchers [24,48,58].

Whereas ``serious game" describes the design of full-fledged games for non-entertainment purposes, ``gamified" applications merely incorporate elements of games (or game ``atoms" [10]). Of course, the boundary between ``game" and ``artifact with game elements" can often be blurry – is Foursquare a game or a ``gamified" application? To complicate matters, this boundary is empirical, subjective and social: Whether you and your friends `play' or `use' Foursquare depends on your (negotiated) focus, perceptions and enactments. The addition of one informal rule or shared goal by a group of users may turn a `merely' `gamified' application into a `full' game. Within game studies, there is an increasing acknowledgement that any definition of `games' has to go beyond properties of the game artifact to include these situated, socially constructed meanings [19,67]. For the present purpose, this means that (a) artifactual as well as social elements of games need to be considered, and (b) artifactual elements should be conceived more in terms of affording gameful interpretations and enactments, rather than being gameful. Indeed, the characteristic of `gamified' applications might be that compared to games, they afford a more fragile, unstable `flicker' of experiences and enactments between playful, gameful, and other, more instrumental-functionalist modes.

% table goes here

As can be seen, this ‘level model’ distinguishes interface design patterns from game design patterns or game mechanics. Although they relate to the shared concept of pattern languages [26], unlike interface design patterns, neither game mechanics nor game design patterns refer to (prototypical) implemented solutions; both can be implemented with many different interface elements. Therefore, they are more abstract and thus treated as distinct.

So to restate, whereas serious games fulfill all necessary and sufficient conditions for being a game, “gamified” applications merely use several design elements from games. Seen from the perspective of the designer, what distinguishes “gamification” from ‘regular’ entertainment games and serious games is that they are built with the intention of a system that includes elements from games, not a full ‘game proper’. From the user perspective, such systems entailing design elements from games can then be enacted and experienced as ‘games proper’, gameful, playful, or otherwise – this instability or openness is what sets them apart from ‘games proper’ for users.

Similar to serious games, “gamification” uses elements of games for purposes other than their normal expected use as part of an entertainment game. Now ‘normal use’ is a socially, historically and culturally contingent category. However, it is reasonable to assume that entertainment currently constitutes the prevalent expected use of games. Likewise, joy of use, engagement, or more generally speaking, improvement of the user experience represent the currently predominant use cases of “gamification” (in the definition proposed in this paper, gameful experiences are the most likely design goal). Still, we explicitly suggest not delimiting “gamification” to specific usage contexts, purposes, or scenarios. Firstly, there are no clear advantages in doing so. Secondly, the murkiness of the discourse on “serious games” can be directly linked to the fact that some authors initially tied the term to the specific context and goal of education and learning, whereas serious games proliferated into all kinds of contexts [61]. Thus, in parallel to Sawyer’s taxonomy of serious games [61], we consider different usage contexts or purposes as potential subcategories: Just as there are training games, health games, or newsgames, there can be gameful design or “gamification” for training, for health, for news, and for other application areas.

To summarize: ``Gamification" refers to
• the use (rather than the extension) of
• design (rather than game-based technology or other game-
related practices)
• elements (rather than full-fledged games)
• characteristic for games (rather than play or playfulness)
• in non-game contexts (regardless of specific usage intentions, contexts, or media of implementation).

This definition contrasts “gamification” against other related concepts via the two dimensions of playing/gaming and parts/whole. Both games and serious games can be differentiated from “gamification” through the parts/whole dimension. Playful design and toys can be differentiated through the playing/gaming dimension (Figure 1). In the broader scheme of trends and concepts identified as related, we find “gamification” or gameful design situated as follows: Within the socio-cultural trend of ludification, there are at least three trajectories relating to video games and HCI: the extension of games (pervasive games), the use of games in non-game contexts, and playful interaction. The use of games in non-game contexts falls into full-fledged games (serious games) and game elements, which can be further differentiated into game technology, game practices, and game design. The latter refers to “gamification” (Figure 2).

\subsection{Game mechanics list}



\subsection{Appropriate game mechanics for the basic project management}

Summarizing the conducted results, the following list of requirements for successful productivity application was made:
“Achievement” game mechanics supports desire to perform tasks
“Pride” game mechanics supports retention of joy from executed tasks
“Avoidance” game mechanics helps a person return to application
“Cascading Information Theory” allows to introduce difficult concepts of GTD (Getting Things Done) and increase a person’s activity in the application
“Communal Discovery” is used in collaborative tasks / task delegation
“Progression Dynamic” helps visualize improvement and progress to a goal.
Techniques used in animation to create believable characters can help to establish an emotional connection to the person

\subsubsection{KPI -- Key Performance Indicator}

\section{Goal commitment formula}

Our lack of adequate emphasis on motivation at work has, in my view, retarded our attempt to maximize performance. In their review of leadership studies, Hogan Curphy and Hogan (1994) found that only about 30 percent of line managers are able to adequately motivate the people who report to them. They imply that in most circumstances, motivation accounts for about half of all performance results.

The late Tom Gilbert, one of the clearest thinkers in performance improvement, was fond of saying that when two people had equal abilities, the enthusiastic member of the pair would achieve about 70 percent more than the unenthusiastic person. Even more troubling is that evidence that a majority of the published studies of organizational development strategies that report measured increases in motivation are fatally flawed (Newman, Edwards and Raju, 1989; Roberts and Robertson, 1992). Strategies that may not work as powerfully or as consistently as claimed include popular employee empowerment strategies, contests, job redesign, leaderless teams and various performance recognition techniques.

If you doubt the importance of motivation in performance, check your answers to the following questions. Why is enthusiastic commitment to work goals so difficult to achieve with many people, even when we pay people well? Why is it onerous, and sometimes impossible, to convince people at work to persist at vital work goals when they encounter interesting but much less important alternative goals? Why do employee reward programs and empowerment strategies sometimes fail or backfire? Do we have to pay people more to get them to work harder? Are people from different cultural backgrounds motivated differently? Is the motivation of “knowledge work” similar to the motivation of “physical work”? Is team motivation different than individual motivation? Why is it that committed people often fail to invest enough effort to fully achieve work goals even though they believe the goals to be important to them and to their organizations? These are some of the questions that trouble human performance consultants.

In the CANE model, motivation is defined as two interlinked processes. The first process leads us to make a commitment to a performance goal and persist the face of distractions from appealing but less important alternative goals. The second motivation process is concerned with the amount and quality of the “mental effort” people invest in achieving the knowledge component of performance goals. These two motivation processes, committed, active and sustained goal pursuit on the one hand, and necessary mental effort to tackle goal-related problems, on the other hand, are the primary motivation goals in the CANE model.

In today’s complex work environments the variety of job tasks that confront all of us change constantly over time. We cannot commit ourselves equally to all tasks. We must prioritize and focus on important tasks in order to be successful. Commitment problems happen when people resist assigning adequate priority to important job tasks. Research on motivation suggests that people with commitment problems may avoid a task altogether and/or argue that the task is less important than some other set of tasks.

Three factors have been found to increase (or decrease) work goal commitment. The first factor is “task assessment”. All of us will analyze any task we are assigned to determine whether we can successfully complete the task. We all tend to ask ourselves two questions about new tasks - “Can I do it?” and “Will I be permitted to do it?”. If we think that we have the ability to accomplish the goal and that we will be permitted to accomplish it, our commitment will increase (Bandura, 1997; Ford, 1992). If we doubt our ability or the organization’s willingness to let us use our skills, commitment will decrease.

Emotion and commitment. The second factor influencing commitment is our mood or emotions. All positive emotions facilitate commitment and all powerful, negative emotions discourage goal commitment (Bower, 1983; 1995; Ford, 1992). This may seem like a minor issue but for temperamental people or in organizations where pressure is high and/or change is constant, negative emotional undercurrents can be strong. Angry or depressed people find it nearly impossible to make a commitment to work goals.

Values and commitment. The final factor that influences the strength of goal commitment is our personal value in the goal. It is my experience that values are the most important element in increasing or decreasing the strength of our commitments. Psychologists now have good evidence that the most important value at work is our belief about whether the achievement of a work goal will increase our personal control or effectiveness (Shapiro et al, 1996; Locke and Latham, 1990). The more we believe that achievement of a work goal will make us more successful, the higher our level of commitment to the goal. The reverse is also true. Few of us will give a high priority to tasks that we sincerely believe will lead us to fail or be perceived as incompetent Utility, Interest and Importance Values.

%CANE Model of Factors Influencing Goal Commitment

Task Assessment Solutions

Solving task assessment problems require that we convince people that they can do a job and that existing barriers to their performance will be removed. Pointing out familiar, past examples of job performance that are similar to the new task helps increase confidence. In addition, job aids can bolster confidence. Involving staff in the elimination of any procedural or policy barriers to performance reduces resistence based on task assessment. The service technicians had excellent job aids which increased their confidence about the form task. The key element here is to persuade or empower people to believe that they can succeed at the task they are avoiding. Bandura (1997) provides extensive examples of solutions in this area.

Mood solutions.

Mood problems often take more time to develop than task assessment or value problems. I find mood problems to be key elements in organizations where a major culture or job change is occurring. This is particularly true in organizations that are changing from a “civil service” to a business culture.

Solutions that have been found to change mood states have included listening to positive mood music; writing or telling about a positive mood-related experience; watching a movie or listening to stories that emphasize positive mood states; and emotion control training through “environmental control strategies” including the choice of how we complete work tasks, adjusting work space and “positive self talk”.

Value solutions

The solution to most commitment problems and opportunities is to convince people that completing the task they are resisting will make them more effective and/or perceived as more effective. People simply will not do what they believe will make them less effective or less successful. Many people are suspicious of change simply because they feel that they will be perceived as less effective under novel, negative or uncertain conditions. They must be convinced that if they commit themselves to the avoided task(s) they will become significantly more effective or successful. The specific solution that accomplishes this goal may be quite different for different individuals and work cultures. Some organizations have adopted various “employee empowerment” solutions to value problems. In many empowerment settings staff are asked to choose their own work goals in order to get them to value their work. There is good research evidence that this is not necessary.

In cases where participatory goal setting is not possible, they find that value for the goal is enhanced if people perceive the goal to be: 1) assigned by a legitimate, trusted authority with an “inspiring vision” that reflects a “convincing rationale” for the goal (importance value), and who; 2) provides expectation of outstanding performance (importance value) and gives: 3) “ownership” to individuals and teams for specific tasks (interest value); 4) expresses confidence in individual and team capabilities (interest value) while; 5) providing feedback on progress that includes recognition for success and supportive but corrective suggestions for mistakes (utility value).

Motivation Solutions

Value problems are often multi-level issues in an organization. In this organization, there were a number of beliefs and patterns that had to be considered. The mangers of the technicians had their own motivational issues to handle. For example, the senior manager acting as sponsor for the motivation project placed a number of constraints on a value solution for the technicians.

Two types of motivation are important at work, persistence and mental effort. Commitment (persistence at a task) is increased by convincing people that: a) the organization will remove unnecessary barriers; b) that achievement of the work goal will make the person more personally effective; and c) that the manager requesting the goal is credible, trustworthy, optimistic (about the person or team’s ability to achieve the goal), able to clearly communicate the vision connected to the goal and willing to give ownership for the accomplishment. Mental effort is enhanced by insuring that the goal assigned is very challenging. Managers must work with people to adjust their confidence level whenever they become over confident (and thus refuse to take responsibility for errors or poor performance) or under confident (and thus find an excuse to procrastinate or avoid the goal altogether).

\section{Motivation in Daniel Pink's ``Drive"}

% WTF?
``THIS IS A BOOK about motivation. I will show that much of what we believe about the subject just isn’t so—and that the insights that Harlow and Deci began uncovering a few decades ago come much closer to the truth. The problem is that most businesses haven’t caught up to this new understanding of what motivates us. Too many organizations—not just companies, but governments and nonprofits as well—still operate from assumptions about human potential and individual performance that are outdated, unexamined, and rooted more in folklore than in science."

% Include a definition of a system here
Societies, like computers, have operating systems—a set of mostly invisible instructions and protocols on which everything runs. The first human operating system—call it Motivation 1.0—was all about survival. Its successor, Motivation 2.0, was built around external rewards and punishments. That worked fine for routine twentieth-century tasks. But in the twenty-first century, Motivation 2.0 is proving incompatible with how we organize what we do, how we think about what we do, and how we do what we do. We need an upgrade.

% When Carrots and Sticks don't work
When carrots and sticks encounter our third drive, strange things begin to happen. Traditional “if-then” rewards can give us less of what we want: They can extinguish intrinsic motivation, diminish performance, crush creativity, and crowd out good behavior. They can also give us more of what we don’t want: They can encourage unethical behavior, create addictions, and foster short-term thinking. These are the bugs in our current operating system.

% When carrots and Sticks work
Carrots and sticks aren’t all bad. They can be effective for rule-based routine tasks— because there’s little intrinsic motivation to undermine and not much creativity to crush. And they can be more effective still if those giving such rewards offer a rationale for why the task is necessary, acknowledge that it’s boring, and allow people autonomy over how they complete it. For nonroutine conceptual tasks, rewards are more perilous— particularly those of the “if-then” variety. But “now that” rewards—noncontingent rewards given after a task is complete—can sometimes be okay for more creative, right- brain work, especially if they provide useful information about performance.

% Include an illustration on how to choose motivation version

% Personality type
Motivation 2.0 depended on and fostered Type X behavior—behavior fueled more by extrinsic desires than intrinsic ones and concerned less with the inherent satisfaction of an activity and more with the external rewards to which an activity leads. Motivation 3.0, the upgrade that’s necessary for the smooth functioning of twenty-first-century business, depends on and fosters Type I behavior. Type I behavior concerns itself less with the external rewards an activity brings and more with the inherent satisfaction of the activity itself. For professional success and personal fulfillment, we need to move ourselves and our colleagues from Type X to Type I. The good news is that Type I’s are made, not born—and Type I behavior leads to stronger performance, greater health, and higher overall well-being.

% Autonomy
Our “default setting” is to be autonomous and self-directed. Unfortunately, circumstances—including outdated notions of “management”—often conspire to change that default setting and turn us from Type I to Type X. To encourage Type I behavior, and the high performance it enables, the first requirement is autonomy. People need autonomy over task (what they do), time (when they do it), team (who they do it with), and technique (how they do it). Companies that offer autonomy, sometimes in radical doses, are outperforming their competitors.

% Mastery
While Motivation 2.0 required compliance, Motivation 3.0 demands engagement. Only engagement can produce mastery—becoming better at something that matters. And the pursuit of mastery, an important but often dormant part of our third drive, has become essential to making one’s way in the economy. Mastery begins with “flow”—optimal experiences when the challenges we face are exquisitely matched to our abilities. Smart workplaces therefore supplement day-to-day activities with “Goldilocks tasks”—not too hard and not too easy. But mastery also abides by three peculiar rules. Mastery is a mindset: It requires the capacity to see your abilities not as finite, but as infinitely improvable. Mastery is a pain: It demands effort, grit, and deliberate practice. And mastery is an asymptote: It’s impossible to fully realize, which makes it simultaneously frustrating and alluring.

% Purpose
Humans, by their nature, seek purpose—a cause greater and more enduring than themselves. But traditional businesses have long considered purpose ornamental—a perfectly nice accessory, so long as it didn’t get in the way of the important things. But that’s changing—thanks in part to the rising tide of aging baby boomers reckoning with their own mortality. In Motivation 3.0, purpose maximization is taking its place alongside profit maximization as an aspiration and a guiding principle. Within organizations, this new “purpose motive” is expressing itself in three ways: in goals that use profit to reach purpose; in words that emphasize more than self-interest; and in policies that allow people to pursue purpose on their own terms. This move to accompany profit maximization with purpose maximization has the potential to rejuvenate our businesses and remake our world.

% Example of L3C integration, proof that things change
For example, in April 2008, Vermont became the first U.S. state to allow a new type of business called the “low-profit limited liability corporation.” Dubbed an L3C, this entity is a corporation—but not as we typically think of it. As one report explained, an L3C “operate[s] like a for-profit business generating at least modest profits, but its primary aim [is] to offer significant social benefits.”

Meanwhile, Nobel Peace Prize winner Muhammad Yunus has begun creating what he calls “social businesses.” These are companies that raise capital, develop products, and sell them in an open market but do so in the service of a larger social mission—or as he puts it, “with the profit-maximization principle replaced by the social-benefit principle.”


% IMPORTANT
Motivation 2.0 suffers from three compatibility problems. It doesn’t mesh with the way many new business models are organizing what we do—because we’re intrinsically motivated purpose maximizers, not only extrinsically motivated profit maximizers. It doesn’t comport with the way that twenty-first-century economics thinks about what we do—because economists are finally realizing that we’re full-fledged human beings, not single-minded economic robots. And perhaps most important, it’s hard to reconcile with much of what we actually do at work—because for growing numbers of people, work is often creative, interesting, and self-directed rather than unrelentingly routine, boring, and other-directed. Taken together, these compatibility problems warn us that something’s gone awry in our motivational operating system.
But in order to figure out exactly what, and as an essential step in fashioning a new one, we need to take a look at the bugs themselves.

% EXPERIMENT ON CHILDREN
One of Lepper and Greene’s early studies (which they carried out with a third colleague, Robert Nisbett) has become a classic in the field and among the most cited articles in the motivation literature. The three researchers watched a classroom of preschoolers for several days and identified the children who chose to spend their “free play” time drawing. Then they fashioned an experiment to test the effect of rewarding an activity these children clearly enjoyed.

The researchers divided the children into three groups. The first was the “expected- award” group. They showed each of these children a “Good Player” certificate—adorned with a blue ribbon and featuring the child’s name—and asked if the child wanted to draw in order to receive the award. The second group was the “unexpected-award” group. Researchers asked these children simply if they wanted to draw. If they decided to, when the session ended, the researchers handed each child one of the “Good Player” certificates. The third group was the “no-award” group. Researchers asked these children if they wanted to draw, but neither promised them a certificate at the beginning nor gave them one at the end.

Two weeks later, back in the classroom, teachers set out paper and markers during the preschool’s free play period while the researchers secretly observed the students. Children previously in the “unexpected-award” and “no-award” groups drew just as much, and with the same relish, as they had before the experiment. But children in the first group—the ones who’d expected and then received an award—showed much less interest and spent much less time drawing. 2 The Sawyer Effect had taken hold. Even two weeks later, those alluring prizes—so common in classrooms and cubicles—had turned play into work.

To be clear, it wasn’t necessarily the rewards themselves that dampened the children’s interest. Remember: When children didn’t expect a reward, receiving one had little impact on their intrinsic motivation. Only contingent rewards—if you do this, then you’ll get that—had the negative effect. Why? “If-then” rewards require people to forfeit some of their autonomy. Like the gentlemen driving carriages for money instead of fun, they’re no longer fully controlling their lives. And that can spring a hole in the bottom of their motivational bucket, draining an activity of its enjoyment.

Lepper and Greene replicated these results in several subsequent experiments with children. As time went on, other researchers found similar results with adults. Over and over again, they discovered that extrinsic rewards—in particular, contingent, expected, “if-then” rewards—snuffed out the third drive.

These insights proved so controversial—after all, they called into question a standard practice of most companies and schools—that in 1999 Deci and two colleagues reanalyzed nearly three decades of studies on the subject to confirm the findings. “Careful consideration of reward effects reported in 128 experiments lead to the conclusion that tangible rewards tend to have a substantially negative effect on intrinsic motivation,” they determined. “When institutions—families, schools, businesses, and athletic teams, for example—focus on the short-term and opt for controlling people’s behavior,” they do considerable long-term damage.

As one leading behavioral science textbook puts it, “People use rewards expecting to gain the benefit of increasing another person’s motivation and behavior, but in so doing, they often incur the unintentional and hidden cost of undermining that person’s intrinsic motivation toward the activity.”

This is one of the most robust findings in social science—and also one of the most ignored. Despite the work of a few skilled and passionate popularizers—in particular, Alfie Kohn, whose prescient 1993 book, Punished by Rewards, lays out a devastating indictment of extrinsic incentives—we persist in trying to motivate people this way.

% GOALS (LOOK DEEPER)
Of course, all goals are not created equal. And—let me emphasize this point—goals and extrinsic rewards aren’t inherently corrupting. But goals are more toxic than Motivation 2.0 recognizes. In fact, the business school professors suggest they should come with their own warning label: Goals may cause systematic problems for organizations due to narrowed focus, unethical behavior, increased risk taking, decreased cooperation, and decreased intrinsic motivation. Use care when applying goals in your organization.

• Offer a rationale for why the task is necessary.A job that’s not inherently interesting can become more meaningful, and therefore more engaging, if it’s part of a larger purpose. Explain why this poster is so important and why sending it out now is critical to your organization’s mission.

• Acknowledge that the task is boring.This is an act of empathy, of course. And the acknowledgment will help people understand why this is the rare instance when “if-then” rewards are part of how your organization operates.

• Allow people to complete the task their own way. Think autonomy, not control. State the outcome you need. But instead of specifying precisely the way to reach it—how each poster must be rolled and how each mailing label must be affixed—give them freedom over how they do the job.

% DO's and DONT'S
Here’s what you shouldn’t do: Offer an “if-then” reward to the design staff. Do not stride into their offices and announce: “If you come up with a poster that rocks my world or that boosts attendance over last year, then you’ll get a ten-percent bonus.” Although that motivational approach is common in organizations all over the world, it’s a recipe for reduced performance. Creating a poster isn’t routine. It requires conceptual, breakthrough, artistic thinking. And as we’ve learned, “if-then” rewards are an ideal way to squash this sort of thinking.
Your best approach is to have already established the conditions of a genuinely motivating environment. The baseline rewards must be sufficient. That is, the team’s basic compensation must be adequate and fair—particularly compared with people doing

Similar work for similar organizations. Your nonprofit must be a congenial place to work. And the people on your team must have autonomy, they must have ample opportunity to pursue mastery, and their daily duties must relate to a larger purpose. If these elements are in place, the best strategy is to provide a sense of urgency and significance—and then get out of the talent’s way.
But you may still be able to boost performance a bit—more for future tasks than for this one—through the delicate use of rewards. Just be careful. Your efforts will backfire unless the rewards you offer meet one essential requirement. And you’ll be on firmer motivational footing if you follow two additional principles.

The essential requirement: Any extrinsic reward should be unexpected and offered only after the task is complete.

Holding out a prize at the beginning of a project—and offering it as a contingency—will inevitably focus people’s attention on obtaining the reward rather than on attacking the problem. But introducing the subject of rewards after the job is done is less risky.
In other words, where “if-then” rewards are a mistake, shift to “now that” rewards—as in “Now that you’ve finished the poster and it turned out so well, I’d like to celebrate by taking you out to lunch.”

As Deci and his colleagues explain, “If tangible rewards are given unexpectedly to people after they have finished a task, the rewards are less likely to be experienced as the reason for doing the task and are thus less likely to be detrimental to intrinsic motivation.”

% DO'S LIST
First, consider nontangible rewards. Praise and positive feedback are much less corrosive than cash and trophies.

Second, provide useful information.

Amabile has found that while controlling extrinsic motivators can clobber creativity, “informational or enabling motivators can be conducive” to it.



\section{Mihaly Csikszentmihalyi's concept of ``Flow"}
\label{sec:flow}

The author has been studying for over 20 years the states of optimal experience--those times when people report feelings of concentration and deep enjoyment. These investigations have revealed that what makes experience genuinely satisfying is a state of consciousness called flow--a state of concentration so focused that it amounts to absolute absorption in an activity. 

Everyone experiences flow from time to time and will recognize its characteristics: people typically feel strong, alert, in effortless control, unselfconscious, and at the peak of their abilities. Both a sense of time and emotional problems seem to disappear, and there is an exhilarating feeling of transcendence. Flow: The Psychology of Optimal Experience describes how this pleasurable state can be controlled, and not just left to chance, by setting ourselves challenges--tasks that are neither too difficult nor too simple for our abilities. With such goals, we learn to order the information that enters consciousness and thereby improve the quality of our lives.

The studies have suggested that the phenomenology of enjoyment has eight major components. When people reflect on how it feels when their experience is most positive, they mention at least one, and often all, of the following:
1. We confront tasks we have a chance of completing;
2. We must be able to concentrate on what we are doing; 3. The task has clear goals;
4. The task provides immediate feedback;
5. One acts with deep, but effortless involvement, that removes from awareness the worries and frustrations
of everyday life;
6. One exercises a sense of control over their actions;
7. Concern for the self disappears, yet, paradoxically
the sense of self emerges stronger after the flow experience is over; and
8. The sense of duration of time is altered.
The combination of all these elements causes a sense of deep enjoyment that is so rewarding people feel that expending a great deal of energy is worthwhile simply to be able to feel it.

A Challenging Activity that Requires Skills
Optimal experiences are reported to occur within sequences of activities that are goal-directed and bounded by rules--activities that require the investment of psychic energy (attention) and that could not be done without skills. Please note that activities do not need to be physical and skills also need not be physical skills. For instance, the most frequently mentioned enjoyable activity the world over was reading, followed closely by being with other people. For those who do not have the right skills, an activity is not challenging; it is simply meaningless. Challenges of competition were found to be stimulating and enjoyable. But when beating the opponent takes precedence in the mind over performing as well as possible, enjoyment tends to disappear. Competition is enjoyable only when it is a means to perfect one's skills; when it becomes an end in itself, it ceases to be fun.

% Include graphic of the flow

The Merging of Action and Awareness
One of the most universal and distinctive features of optimal experience is the people become so involved in what they are doing that the activity becomes spontaneous, almost automatic; they stop being aware of themselves as separate from the actions they are performing. It often requires strenous physical exertion, or highly disciplined mental activity to enter a continuous flow.

Clear Goals and Feedback
Unless a person learns to set goals and to recognize and gauge feedback in their activities, she will not enjoy them. For activities that are creative or open-ended in nature, a person must develop a strong sense of what she intends to do or negotiate goals and rules during the activity. These goals and rules provide benchmarks for feedback. The kind of feedback we work toward is in, and of itself, often unimportant. What makes feedback valuable is the symbolic message it contains: that I have succeeded in my goal.

Concentration on the Task at Hand
One of the most frequently mentioned dimensions of the flow experience is that, while it lasts, one is able to forget all the unpleasant aspects of life. The task requires such concentration that only a very select range of information can be allowed into awareness.

The Paradox of Control
The flow experience is typically described as involving a sense of control--or more precisely, as lacking the sense of worry about losing control that is typical in many situations of normal life. What people enjoy is not the sense of being in control, but the sense of exercising control in difficult situations. However, when a person becomes dependent on the ability to control an enjoyable activity then he loses the ultimate control: the freedom to determine the content of consciousness. While experiences are capable of improving the quality of existence by creating order in the mind, they can also become addictive, at which point the self becomes captive of a certain kind of order, and is then unwilling to cope with the ambiguities of life.

The Loss of Self-Consciousness
When in a flow experience, what slips below the threshold of awareness is the concept of self, the information we use to represent to ourselves who we are. And being able to forget temporarily who we are seems to be very enjoyable. When not preoccupied with our selves, we actually have a chance to expand the concept of who we are. Loss of self-consciousness can lead to self-transcendence, to a feeling that the boundaries of our being have been pushed forward.

The Transformation of Time
One of the most common descriptions of optimal experience is that time no longer seems to pass the way it ordinarily does. Generally, after the experience we do not know where the time went; however, during the actual experience, time seems to stand still.
The key element of an optimal experience is that it is an end in itself. It is an autotelic experience. The term "autotelic" derives from two Greek words, "auto" meaning self, and "telos" meaning goal. It refers to a self-contained activity, one that is done not with the expectation of some future benefit, but simply because the doing itself is the reward. Teaching children in order to turn them into good citizens is not autotelic, whereas teaching them because one enjoys interacting with children is. Most enjoyable activities are not natural; they demand an effort that initially one is reluctant to make. But once the interaction starts to provide feedback to the person's skills, it usually begins to be intrinsically rewarding.
Flow in the family context has five characteristics:
•Clarity: children know what parents expect from them; •Centering: children know that their parents are interested in what they are doing in the present;
•Choice: children feel that they have a variety of possibilities from which to choose;
•Commitment: trust that allows the child to feel comfortable enough to set aside the shield of defenses and become unself-consciously involved; and
•Challenge: providing increasingly complex opportunities for action.

%\section{Reconsidering gamification for productivity}

\section{General guidelines by XXXXXX}

% Scheme of the effectiveness
