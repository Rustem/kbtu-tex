\chapter{Building productivity mobile application}

In this chapter the prototype of personal performance mobile application will be considered.

\section{Abstractions and games}
There is an important notion of feedback within the concept of ``Flow" and appearing in all of the method of project management.

The only application from considered ones in chapter \ref{chap:apps} did try to use a concept of feedback: ``Carrot". It used a game mechanic called ``Negative reinforcement" in order to state a bad system condition: there is a lot of tasks, that require attention.

\subsection{Concept of time limitation}

Each task (that has a deadline) can be represented as a system with dynamic complexity. As focus is a limited resource, as a time, tasks that don't get enough attention can lead to a bad state of a system very fast. Thus causing lack of time and lack of focus, which recursively make situation even worse.

Listed application didn't use the concept of limited resource. One of the ways to illustrate the limited resource was introduced by popular games: bubbles and tetris.

Using a mobile screen as limitation to available tasks, it is possible to construct the following abstraction:

\begin{compactitem}
\item The vertical axis represents time available till deadline
\item The space represents required focus and illustrates which tasks require immediate action
\item Each can be represented as a circle, initial (normal) size of the circle is determined by its importance (important or not)
\item Space can also group related tasks
\end{compactitem}


\section{Game mechanics in use}

In the case of application there are two main dynamics used:

\begin{compactenum}
\item Appointment Dynamic, which requires to return at a predefined time to mark a task as ``in progress" or ``complete".\item Using disincentives to illustrate limited space: if there is not enough space, it is impossible to add new task
\end{compactenum}