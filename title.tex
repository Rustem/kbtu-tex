% Title example
\begin{titlepage}
    \pagestyle{empty}
    \begin{center}
        {\bf{\MakeUppercase{Ministry of education and science of the republic of Kazakhstan}}

        \vspace{14pt}

        JSC ``Kazakh-British Technical University"\\
        Faculty of Information Technologies}
       
       \vspace{14pt}
       
        % Admitted to defence
        \begin{flushright}
            {\bf \MakeUppercase{``Admitted to defence"}}

            Chair of CE Department: {\em Lyazzat B. Atymtayeva}\\
            \vspace{0.5\baselineskip}
            \rule{13em}{0.4pt}\\
            \vspace{14pt}
          Writer of dissertation: {\em Vadim V. Kotov}
          \vspace{14pt}
        \end{flushright}
        
        {\bf
        \MakeUppercase{Master's Dissertation}\\
        6M070300 -- ``Information Systems" specialty}

        \vspace{14pt}

        Theme: {\bf ``Game Mechanics for Stimulating High Performance of Project Participants"}
        
        \vspace{28pt}
        
        \begin{figure}[ht]
            \begin{minipage}[t]{0.6\linewidth}
                {\bf Scientific supervisor}\\

                {\em Timur F. Umarov,\\
                Ph.D. Computer Science,\\
                Associate professor}\\
            \end{minipage}
        \end{figure}

    \end{center}


    \begin{center}
        \vfill
        Almaty, 2013
    \end{center}

    \pagebreak
    % End title page

    % Start all other stuff

    \begin{center}
    {\bf{\MakeUppercase{Ministry of education and science of the republic of Kazakhstan}}

        \vspace{14pt}

        JSC ``Kazakh-British Technical University"\\
        Faculty of Information Technologies}

        \vspace{14pt}
        
        \begin{flushright}
            {\bf \MakeUppercase{``Approved by"}}

            Chair of CE Department\\
            Lyazzat B. Atymtayeva,\\
            Doctor of physical and mathematical sciences,\\
            Professor\\

            \vspace{0.5\baselineskip}
            ``\rule{2em}{0.4pt}" \rule{8em}{0.4pt} 2013\\
            \end{flushright}
        
        {\bf
        \MakeUppercase{Assignment for master's dissertation}}

    \end{center}

    \setlength{\parindent}{0pt}
    \setlength{\parskip}{1ex plus 0.5ex minus 0.2ex}

    V. Kotov

    {\bf 6M070300 -- Information Systems}

    \emph{Theme:} ``Game Mechanics for Stimulating High Performance of Project Participants"
    
   \emph{Objectives:}
    Explore modern project management methodologies and gamification theory in order to understand elements affecting performance of project participants and provide guidelines in order to increase their effectiveness.
    
    \emph{Source data:}
    
    \begin{compactitem}
    \item Sketch (Bohemian Coding)
    \item Processing (\url{http://processing.org/})
    \end{compactitem}


    \begin{figure}[ht]
            \begin{minipage}[t]{0.6\linewidth}
                {\bf Scientific supervisor}\\

                {\em Timur F. Umarov,\\
                Ph.D. Computer Science,\\
                Associate professor}\\
                
                ``\rule{2em}{0.4pt}" \rule{8em}{0.4pt} 2013\\
         \end{minipage}
    \end{figure}
    
    \begin{center}
        \vfill
        Almaty, 2013
    \end{center}

    \pagebreak

    \begin{centering}
        {\bf{\MakeUppercase{Ministry of education and science of the republic of Kazakhstan}}

        \vspace{14pt}

        JSC ``Kazakh-British Technical University"\\
        Faculty of Information Technologies}

        \vspace{14pt}
        
        {\bf
        \MakeUppercase{Scientific publications}\\
        by Information Systems M.Sc. programme student Vadim Kotov
        }

        \vspace{14pt}
    \end{centering}

    \begin{centering}

        \begin{longtable}{|l|p{0.25\textwidth}|p{0.25\textwidth}|l|l|}
            \hline
            \textbf{No.} & \textbf{Title} & \textbf{Publisher} & \textbf{Pages} & \textbf{Co-authors}\\
            \endhead

            \hline
            1 & 
Productivity mobile applications analysis: game-mechanics and empathy & \small KBTU, Сборник трудов, Международный Конкурс студенческих проектов по информационным технологиям Алматы, 2013 & & \\
            \hline
            2 & Productivity mobile application: information architecture and prototyping & \small KBTU, Сборник трудов, Международный Конкурс студенческих проектов по информационным технологиям Алматы, 2013 & & \\
            \hline

            \end{longtable}
        \end{centering}

        \begin{figure}[ht]
            \begin{minipage}[t]{0.5\linewidth}
                Author\\

                \rule{13em}{0.4pt}\\
                Vadim Kotov\\
            \end{minipage}
            \begin{minipage}[t]{0.5\linewidth}
                Scientific secretary\\

                \rule{13em}{0.4pt}\\
            \end{minipage}
        \end{figure}
        
        \pagebreak
        \pagebreak

    \begin{centering}
        {\bf{\MakeUppercase{Ministry of education and science of the republic of Kazakhstan}}

        \vspace{14pt}

        JSC ``Kazakh-British Technical University"\\
        Faculty of Information Technologies}

        \vspace{14pt}
        
        {\bf
        \MakeUppercase{Scientific publications}\\
        by Information Systems M.Sc. programme student Rustem Kamun
        }

        \vspace{14pt}
    \end{centering}

    \begin{centering}

        \begin{longtable}{|l|p{0.25\textwidth}|p{0.25\textwidth}|l|l|}
            \hline
            \textbf{No.} & \textbf{Title} & \textbf{Publisher} & \textbf{Pages} & \textbf{Co-authors}\\
            \endhead
            \hline
            1 & Smart Shopping Cart & \small KBTU, Сборник трудов, Международный Конкурс студенческих проектов по информационным технологиям Алматы, 28 апреля, 2012 & 153-159 & \\
            \hline
            2 & Synergy of Service-Oriented Architecture and Cloud Computing Introduction & \small KBTU, Сборник трудов, Международный Конкурс студенческих проектов по информационным технологиям Алматы, 2013 & & \\
            \hline            
            \end{longtable}
        \end{centering}

        \begin{figure}[ht]
            \begin{minipage}[t]{0.5\linewidth}
                Author\\

                \rule{13em}{0.4pt}\\
                Rustem Kamun\\
            \end{minipage}
            \begin{minipage}[t]{0.5\linewidth}
                Scientific secretary\\

                \rule{13em}{0.4pt}\\
            \end{minipage}
        \end{figure}
        
        \pagebreak

    \begin{centering}
        {\bf{\MakeUppercase{Ministry of education and science of the republic of Kazakhstan}}

        \vspace{14pt}

        JSC ``Kazakh-British Technical University"\\
        Department of Computer Engineering}
       \vspace{14pt}

        {\bf
        Dissertation supervisor review\\
        by Timur Umarov
        }
        
        on ``Game Mechanics for Stimulating High Performance of Project Participants"

        \vspace{14pt}
        \small 6M070300 -- Information Systems

    \end{centering}
    
    
    There is an ongoing problem of low performance and motivation at work. The objective of the dissertation was to explore modern project management methodologies and gamification theory in order to understand elements affecting performance of project participants and provide guidelines in order to increase their effectiveness.
    
    Results of the research allowed to prototype a mobile productivity application, that uses the notion os ``flow" and has effective feedback mechanisms.
    
    The author of the master dissertation did a solid work, and constructed a good foundation on further work in the area of system thinking, motivation and productivity.
    
    I recommend to evaluate the dissertation of Vadim V. Kotov as ``excellent" and award him with a deserved qualification ``MSc in Information Systems" by specialty 6M070300.
    

         \begin{figure}[ht]
            \begin{minipage}[t]{0.6\linewidth}
                {\bf Scientific supervisor}\\

                {\em Timur F. Umarov,\\
                Ph.D. Computer Science,\\
                Associate professor}\\
                
                ``\rule{2em}{0.4pt}" \rule{8em}{0.4pt} 2013\\
         \end{minipage}
    \end{figure}
        
        \pagebreak
        

    \begin{centering}
        {\bf{\MakeUppercase{Ministry of education and science of the republic of Kazakhstan}}

        \vspace{14pt}

        JSC ``Kazakh-British Technical University"\\
        Department of Computer Engineering}
       \vspace{14pt}

        {\bf
        Dissertation supervisor review\\
        by Timur Umarov
        }
        
        on ``Ensuring faultless web services specifications by developing discrete and modular systems blueprints"

        \vspace{14pt}
        \small 6M070300 -- Information Systems

    \end{centering}
    
    
    
 Rustem A. Kamun have made a solid work by researching business processes and their interaction in Web environment. I am sure that Rustem Kamun’s result is very valuable and actual for the Internet computing and Web services.
    
He carefully analyzed the issues of existing technologies, the issues of existing standards like WS-CDL and BPEL-WS and proposed the formal theory based on session typing.
    
Rustem A. Kamun confirmed the feasibility of the proposed theory by applying it in designing business protocols. He also received significant results by comparing the performance of Session Java with existing technologies.
     
It’s worth noting that Rustem's work has a huge impact on Web-service analysis and made a base for his future Ph.D.
I recommend to evaluate the dissertation of Rustem A. Kamun as ``excellent" and award him with a deserved qualification ``MSc in Information Systems" by specialty 6M070300.
    

         \begin{figure}[ht]
            \begin{minipage}[t]{0.6\linewidth}
                {\bf Scientific supervisor}\\

                {\em Timur F. Umarov,\\
                Ph.D. Computer Science,\\
                Associate professor}\\
                
                ``\rule{2em}{0.4pt}" \rule{8em}{0.4pt} 2013\\
         \end{minipage}
    \end{figure}
        
        \pagebreak

    \begin{centering}
        {\bf{\MakeUppercase{Ministry of education and science of the republic of Kazakhstan}}}

       \vspace{14pt}

        {\bf
        Opponent's review of the Master's Dissertation\\
        }

        \vspace{14pt}
        
        {\bf``Game Mechanics for Stimulating High Performance of Project Participants"}\\
        {\small Kazakh-British Technical University\\
        6M070300 -- Information Systems\\
        Vadim V. Kotov\\}
        
        \vspace{14pt}
        
    \end{centering}
    
    Master's Dissertation of Vadim Kotov explores several project management methodologies (known as iterative) and gamification theory in order to understand elements affecting performance of project participants and provide guidelines in order to increase their effectiveness.
    
    This Dissertation introduces a complex overview of causes of emerging iterative methodologies, motivation management and the concept of ``Flow". The resulting application prototype incorporates a very different approach and focuses on feedback.
    
    Dissertation consists of 5 chapters, presenting the necessary background combination of Project Management methodologies and psychological mechanisms behind motivation.
    
    In addition I wish to say that it is a very solid work. Results from the research could be used to stimulate not only project participants' performance, but also personal performance. The dissertation is a good starting point for many further researches on system thinking in project management.
    
    I recommend to evaluate the dissertation of Vadim V. Kotov as ``excellent" and award him with a deserved qualification ``MSc in Information Systems" by specialty 6M070300.

        \begin{figure}[ht]
            \begin{minipage}[t]{0.7\linewidth}{\em \small
                {\bf Opponent: }\\
                Maksat Maratov (M.Sc. in Information Systems)\\
                Lecturer, Department of Information Technology\\
                International Information Technology University
                \vspace{14pt}
                
                \rule{13em}{0.4pt}\\
                }
            \end{minipage}
        \end{figure}
        
    \end{titlepage}